\chapter{Universitas Alkimia Batin} %192

\bahasa
26 Februari 1979 pagi di Aula Buddha

\english
26 February 1979 am in Buddha Hall

\bahasa
Pertanyaan pertama:

\english
The first question:

\bahasa
OSHO TERKASIH,
YA DAN TIDAK!

\english
BELOVED OSHO,
YES AND NO!

\bahasa
Prem Madhu, manusia adalah dilema, dia keduanya ya dan tidak. Itu tidak normal di dalam dirimu, itu adalah keadaan normal manusia. Manusia adalah setengah bumi, setengah langit; sebagian materi, sebagian kesadaran; sebagian debu, sebagian ilahi. Manusia adalah ketegangan. Friedrich Nietzsche mengatakan "sebuah tali membentang di antara dua tak terbatas".

\english
Prem Madhu, man is a dilemma, he is both yes and no. It is not abnormal in you, it is the normal state of humankind. Man is half earth, half sky; part matter, part consciousness; part dust, part divine. Man is a tension. Friedrich Nietzsche says "a rope stretched between two infinities".

\bahasa
Masa lalu adalah hewani dan masa depan adalah Tuhan. Dan di antara keduanya adalah manusia -- setengah hewan, setengah malaikat. Tidak datang dari masa lalu, ya adalah kemungkinan untuk masa depan. Keraguan berasal dari kegelapan, kepercayaan-penuh adalah produk sampingan dari cahaya. Diri-sejati yang tinggi di dalam dirimu adalah selalu percaya-penuh, yang lebih rendah adalah licik dan selalu meragukan. Dan engkau keduanya, itulah dirimu sekarang.

\english
The past is that of an animal and the future is that of God. And between the two is man -- half animal, half angel. The no comes from the past, the yes is a possibility for the future. Doubt comes from darkness, trust is the by-product of light. The higher self in you is always trusting, the lower is cunning and always doubting. And you are both, as you are now.

\bahasa
Manusia secara alami skizofrenia. Skizofrenia bukan penyakit; itu ukan patologi, memang begitu keadaan manusia normal. Itu mulai terlihat seperti patologi hanya jika mengarah ke ekstrem, ketika ya dan tidak terbagi sehingga bahkan tidak ada "dan" menjembatani mereka. Ketika mereka menjadi tidak terjembatani, maka itu menjadi patologis. Jika tidak, setiap manusia selalu memiliki dualitas, dalam keadaan salah-satu/atau. Tidak ada hewan lain yang berada dalam keadaan itu. Anjing hanyalah anjing, dan singa adalah singa, dan pepohonan adalah pepohonan, dan bebatuan adalah bebatuan. Mereka tidak memiliki dualitas apapun, tidak ada pembagian.

\english
Man is naturally schizophrenic. Schizophrenia is not a disease; it is not pathology, it is the state of normal human beings. It starts looking like a pathology only when it goes to the extreme, when yes and no are so divided that there is not even an "and" to bridge them. When they become unbridgeable, then it becomes pathological. Otherwise every human being is always in a kind of duality, in a state of either/or. No other animal is in that state. Dogs are simply dogs, and lions are lions, and trees are trees, and rocks are rocks. They don't have any duality, there is no division.

\bahasa
Manusia itu dualis, ganda, terbagi. Itu adalah kesengsaraannya, tapi juga kemungkinan untuk kebahagiaan-sejatinya. Itu adalah penderitaannya, namun diluar dari penderitaan ini sukacita besar bisa lahir. Tidak ada hewan yang bisa menjadi penuh sukacita kecuali manusia. Pernahkah engkau melihat hewan yang penuh sukacita -- penuh sukacita seperti buddha, Ramakrishna? Tidak mungkin menemukan binatang yang penuh dengan sukacita. Bahkan bunga mawar diantara dengan begitu banyak bunga yang indah tidak dalam penuh sukacita seperti Yesus. Mawar hanyalah sekuntum mawar; tidak ada kegembiraan, tidak ada yang meluap, tidak ada sukacita. Ini adalah masalah fakta -- bukan berarti sesuatu yang luar biasa sedang terjadi, bukan berarti sesuatu dari luar telah turun, bukan berarti Tuhan telah disadari, bukan berarti cahaya telah datang dan menembus ke inti terdalam dari keberadaanmu dan engkau penuh dengan itu dan engkau tercerahkan.

\english
Man is dual, double, divided. It is his misery, but it is also the possibility for his bliss. It is his agony, but out of this agony ecstasy can be born. No animal can be ecstatic except man. Have you seen any animal ecstatic -- ecstasy like a buddha, a Ramakrishna? There is no possibility of coming across an animal which is so ecstatic. Even the rosebush with so many beautiful flowers is not ecstatic in the sense Jesus is. The rosebush is simply a rosebush; there is no exuberance, there is no overflowing, there is no rejoicing. It is a matter of fact -- not that something incredible is happening, not that something from the beyond has descended, not that God has been realized, not that light has come and penetrated to the deepest core of your being and you are full of it and you are enlightened.

\bahasa
Burung di sayapnya bebas, tapi tidak tahu apa-apa tentang kebebasan. Hanya manusia, meski dia mungkin dipenjarakan, tahu tentang kebebasan. Karenanya ada kesengsaraan; perbudakan di satu sisi, dan visi kebebasan di sisi lain. Kenyataan, kenyataan yang buruk, dan kemungkinan yang sangat bercahaya.

\english
The bird on the wing is free, but knows nothing about freedom. Only man, even though he may be imprisoned, knows about freedom. Hence the misery; the bondage on one hand, and the vision of freedom on the other. The reality, the ugly reality, and the tremendously luminous possibility.

\bahasa
Manusia bisa menderita ketika tidak ada hewan lain yang bisa menderita. Pernahkah engkau melihat ada hewan yang menangis di hatinya, menangis, bunuh diri? Pernahkah engkau melihat hewan tertawa, sebuah tawa perut yang mengguncang fondasi? Tidak, semua hal ini mungkin hanya untuk manusia. Karenanya kemegahan manusia ada, karenanya martabatnya ada, dan karenanya kecemasan miliknya juga ada.

\english
Man can be miserable as no other animal can ever be miserable. Have you seen any animal crying its heart out, weeping, committing suicide? Have you seen any animal laughing, a belly laughter that shakes the very foundations? No, all these things are possible only for man. Hence the grandeur of man, hence his dignity, and hence his anxiety too.

\bahasa
Kecemasan berakhir apakan engkau akan berhasil atau tidak, apakh waktu ini akan terjadi atau tidak. Kecemasan adalah konsekuensi alami dari dua kemungkinan yang berlawanan: seseorang bisa jatuh ke neraka, dan seseorang dapat naik ke surga.

\english
The anxiety is over whether you are going to make it or not, whether this time it is going to happen or not. The anxiety is a natural consequence of two diametrically opposite possibilities: one can fall into hell, and one can rise into heaven.

\bahasa
Manusia hanyalah sebuah tangga, dan engkau bergerak di tangga ini seperti yo-yo. Satu saat engkau berada di surga, saat lain engkau berada di neraka. Satu saat tiba-tiba berada di puncak yang diterangi sinar matahari, saat lain berada di lembah paling gelap yang pernah engkau temui. Satu saat cinta, berbagi; saat lain kemarahan, keputusasaan. Satu saat seperti hati yang meluas sehingga engkau bisa menampung seluruh dunia, dan saat lain engkau begitu kejam sehingga engkau tidak dapat membayangkan bahwa engkau memiliki kemungkinan untuk bersikap begitu jahat. Manusia terus bergerak di antara kedua yang tak terbatas ini terus-menerus seperti pendulum.

\english
Man is just a ladder, and you move on this ladder like a yo-yo. One moment you are in heaven, another moment you are in hell. One moment suddenly the sunlit peak, another moment the darkest valley that you have ever come across. One moment love, sharing; another moment anger, miserliness. One moment such an expanded heart that you can contain the whole world, and another moment you are so mean that you cannot imagine you had this possibility to be so mean. Man goes on moving between these two infinities continuously like a pendulum.

\bahasa
Prem Madhu, pertanyaanmu begitu berarti, karena itu pertanyaan semua orang. Itu bukan sebuah pertanyaan, itu jauh lebih nyata. Itu adalah sebuah masalah; tidak ada jawaban yang bisa membantu, beberapa solusi harus dicari.

\english
Prem Madhu, your question is significant, because it is everybody's question. It is not a question, it is far more existential. It is a problem; no answer can help, some solution has to be searched for.

\bahasa
Sekarang, ada dua kemungkinan untuk solusinya. Pertama adalah untuk jatuh kembali dan puas dengan hewanimu. Menjadi puas -- itulah yang coba dilakukan jutaan orang; minum, makan, tidur, dan melupakan semua tentang tantangan hidup yang lebih besar. Makan, minum dan menikah, karena besok kita tidak akan ada lagi. Itulah yang dikatakan oleh kaum materialis.

\english
Now, there are two possibilities for the solution. One is to fall back and be satisfied with your animalness. Be satisfied -- that's what millions are trying to do; drink, eat, sleep, and forget all about the greater challenges of life. Eat, drink and be merry, because tomorrow we shall be no more. That's what the materialist says.

\bahasa
Kaum materialis telah menerima diri yang lebih rendah; dia telah menyangkal diri yang lebih tinggi hanya untuk bela diri belaka. Dia tidak menyangkalnya karena dia tahu itu tidak ada; tidak -- dia tidak tahu apa-apa tentang hal itu. Dia telah menyangkal hal itu karena jika tidak menyangkalnya maka baik salah-satu/atau akan terbuka lagi. Lagi-lagi seseorang dalam masalah, lagi-lagi sesuatu harus dilakukan, lagi-lagi kenyamanan menghilang. Lagi-lagi perjalanan, pengembaraan, dan ketidaknyamanan dan ketidaksenangan dan ketidakamanan perjalanan.

\english
The materialist has accepted the lower self; he denies the higher self just in sheer self defense. He does not deny it because he knows that it is not; no -- he knows nothing about it. He denies it because if he does not deny it then that either/or opens up again. Again one is in a problem, again something has to be done, again the at-easeness is lost. Again the journey, the wandering, and the discomfort and the inconvenience and the insecurity of the journey.

\bahasa
Lebih baik mengatakan bahwa yang lebih tinggi tidak ada, bahwa tidak ada Tuhan, bahwa tidak pernah ada Tuhan, bahwa tidak ada jiwa, bahwa tidak ada batin, manusia itu tidak memiliki interioritas, manusia itu hanya apa yang tampak dari luar, bahwa manusia adalah perilakunya dan tidak ada jiwa di dalam dirinya.

\english
It is better to say that the higher does not exist, that there is no God, that there has never been any God, that there is no soul, that there is nothing inner, that man has no interiority, that man is just what he is from the outside, that man is his behavior and there is no soul in him.



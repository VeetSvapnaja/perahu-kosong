\chapter{Jiwa adalah sebuah Pertanyaan} %348

\bahasa
Pertanyaan Pertama:

\english
The first question:

\bahasa
OSHO TERKASIH,\\
KENAPA BEGITU SULIT UNTUK BERHUBUNGAN ?

\english
BELOVED OSHO,\\
WHY IS IT SO DIFFICULT TO RELATE?

\bahasa
Deva Shanta, karena engkau belum ada. Ada kekosongan batin dan ketakutan bahwa jika engkau berhubungan dengan seseorang, cepat atau lambat engkau akan terlihat kosong. Oleh karena itu tampaknya lebih aman untuk menjaga jarak dengan orang-orang; paling tidak engkau dapat berpura-pura ada.

\english
Deva Shanta, because you are not yet. There is an inner emptiness and the fear that if you relate with somebody, sooner or later you will be exposed as empty. Hence it seems safer to keep a distance with people; at least you can pretend you are.

\bahasa
Engkau belum ada. Engkau belum lahir, engkau hanyalah sebuah kesempatan. Engkau belum menjadi sebuah pemenuhan - dan hanya dua orang terpenuhi yang dapat berhubungan. Untuk berhubungan adalah salah satu yang terbesar dalam hidup: berhubungan berarti mencintai, berhubungan berarti berbagi. Tapi sebelum dapat berbagi, engkau harus mempunyai. Dan sebelum engkau dapat mencintai engkau harus penuh cinta, berlimpah dengan cinta.

\english
You are not. You are not yet born, you are only an opportunity. You are not yet a fulfillment -- and only two fulfilled persons can relate. To relate is one of the greatest things of life: to relate means to love, to relate means to share. But before you can share, you must have. And before you can love you must be full of love, overflowing with love.

\bahasa
Dua benih tidak dapat berhubungan, keduanya tertutup. Dua bunga bisa berhubungan; mereka terbuka, mereka dapat mengirim wewangian mereka satu sama lain, mereka dapat menari di bawah sinar matahari yang sama dan dengan angin yang sama, mereka bisa berdialog, mereka bisa berbisik. Tapi itu tidak mungkin untuk dua benih. Benih benar-benar tertutup, tanpa jendela - bagaimana caranya berhubungan?

\english
Two seeds cannot relate, they are closed. Two flowers can relate; they are open, they can send their fragrances to each other, they can dance in the same sun and in the same wind, they can have a dialogue, they can whisper. But that is not possible for two seeds. Seeds are utterly closed, windowless -- how to relate?

\bahasa
Dan itulah situasinya. Manusia dilahirkan sebagai sebuah benih; dia dapat menjadi sekuntum bunga, mungkin dia tidak. Semuanya tergantung padamu, apa yang engkau lakukan dengan dirimu sendiri; itu semua tergantung padamu apakah engkau bertumbuh atau tidak. Itu adalah pilihanmu - dan setiap saat pilihan harus dihadapi; setiap saat engkau berada di persimpangan jalan.

\english
And that is the situation. Man is born as a seed; he can become a flower, he may not. It all depends on you, what you do with yourself; it all depends on you whether you grow or you don't. It is your choice -- and each moment the choice has to be faced; each moment you are on the crossroads.

\bahasa
Jutaan orang memutuskan untuk tidak bertumbuh. Mereka tetap menjadi benih; mereka tetap potensi, mereka tidak pernah menjadi kenyataan. Mereka tidak tahu realisasi diri, mereka tidak tahu aktualisasi-diri, mereka tidak tahu apa-apa tentang keberadaan. Benar-benar kosong mereka hidup, benar-benar kosong mereka mati. Bagaimana mereka dapat berhubungan?

\english
Millions of people decide not to grow. They remain seeds; they remain potentialities, they never become actualities. They don't know what self-realization is, they don't know what self-actualization is, they don't know anything of being. Utterly empty they live, utterly empty they die. How can they relate?

\bahasa
Itu akan mengekspos dirimu -- ketelanjanganmu, keburukanmu, kekosonganmu -- lebih aman, sepertinya, untuk menjaga jarak. Bahkan kekasih tetap menjaga jarak; Mereka datang hanya sejauh ini, dan mereka tetap waspada kapan harus kembali. Mereka memiliki batasan-batas; mereka tidak pernah melewati batas-batas, mereka tetap terbatas pada batas-batas mereka.

\english
It will be exposing yourself -- your nudity, your ugliness, your emptiness -- safer, it seems, to keep a distance. Even lovers keep distance; they come only so far, and they remain alert to when to turn back. They have boundaries; they never cross the boundaries, they remain confined to their boundaries.

\bahasa
Ya, ada semacam hubungan, tapi itu bukan hubungan-keterkaitan, itu adalah kepemilikan: suami memiliki istri, istri memiliki suami, orang tua memiliki anak, dan seterusnya dan seterusnya. Tapi untuk memiliki bukan untuk berhubungan. Sebenarnya untuk memiliki adalah untuk menghancurkan semua kemungkinan hubungan. Jika engkau berhubungan, engkau menghormati; engkau tidak dapat memiliki. Jika engkau berhubungan, ada penghormatan yang besar. Jika engkau berhubungan, engkau datang sangat dekat, sangat sangat dekat, dalam keintiman yang mendalam, tumpang tindih. Namun kebebasan orang lain tidak ikut dicampuri, orang lain tetap menjadi individu yang independen. Hubungannya adalah hubungan aku/engkau bukan aku/itu -- tumpang tindih, saling bergabung bersama, namun dalam arti independen.

\english
Yes, there is a kind of relationship, but it is not that of relating, it is that of possession: the husband possesses the wife, the wife possesses the husband, the parents possess the children, and so on and so forth. But to possess is not to relate. In fact to possess is to destroy all possibilities of relating. If you relate, you respect; you cannot possess. If you relate, there is great reverence. If you relate, you come very close, very very close, in deep intimacy, overlapping. Still the other's freedom is not interfered with, still the other remains an independent individual. The relationship is that of I/thou, not that of I/it -- overlapping, interpenetrating, yet in a sense independent.

\bahasa
Khalil Gibran berkata: "Jadilah seperti dua tiang yang menopang atap yang sama, tapi jangan mulai memiliki yang lain, tinggalkan yang lainnya independen. Menopang atap yang sama -- atap itu adalah cinta."

\english
Khalil Gibran says: "Be like two pillars that support the same roof, but don't start possessing the other, leave the other independent. Support the same roof -- that roof is love."

\bahasa
Dua kekasih mendukung sesuatu yang tak terlihat dan sesuatu yang sangat berharga: beberapa puisi tentang keberadaan, beberapa musik terdengar di dalam relung terdalam keberadaan mereka. Mereka mendukung keduanya, mereka mendukung beberapa harmoni, namun tetap saja mereka tetap independen. Mereka dapat mengekspos diri mereka ke yang lainnya, karena tidak ada rasa takut. Mereka tahu mereka ADA. Mereka tahu keindahan batin mereka, mereka tahu parfum batin mereka; tidak ada ketakutan

\english
Two lovers support something invisible and something immensely valuable: some poetry of being, some music heard in the deepest recesses of their existence. They support both, they support some harmony, but still they remain independent. They can expose themselves to the other, because there is no fear. They know they ARE. They know their inner beauty, they know their inner perfume; there is no fear.

\bahasa
Tapi biasanya rasa takut itu ada, karena engkau tidak punya parfum. Jika engkau mengekspos dirimu engkau hanya akan berbau busuk. Engkau akan berbau kecemburuan, kekebencian, kemarahan, nafsu. Engkau tidak akan memiliki parfum cinta, doa, welas asih.

\english
But ordinarily the fear exists, because you don't have any perfume. If you expose yourself you will simply stink. You will stink of jealousies, hatreds, angers, lust. You will not have the perfume of love, prayer, compassion.

\bahasa
Jutaan orang telah memutuskan untuk tetap menjadi benih. Mengapa? Ketika mereka dapat menjadi bunga dan mereka juga dapat menari di atas angin, matahari dan bulan, mengapa mereka memutuskan untuk tetap menjadi benih? Ada sesuatu dalam keputusan mereka: benih lebih aman daripada bunga. Bunga itu rapuh; benih tidak rapuh, benih terlihat lebih kuat. Bunga itu dapat dimusnahkan dengan sangat mudah; hanya angin kencang dan kelopak bunga akan melayang.

\english
Millions of people have decided to remain seeds. Why? When they can become flowers and they can also have a dance in the wind and the sun and the moon, why have they decided to remain seeds? There is something in their decision: the seed is more secure than the flower. The flower is fragile; the seed is not fragile, the seed looks stronger. The flower can be destroyed very easily; just a strong wind and the petals will blow away.

\bahasa
Benih tidak dapat dimusnahkan begitu mudah oleh angin, benih sangat terlindungi, aman. Bunga itu terbuka -- hal yang begitu lemah, dan terekpos oleh begitu banyak bahaya: angin kencang mungkin datang, mungkin hujan, kucing dan anjing, matahari mungkin terlalu panas, beberapa manusia bodoh dapat memetik bunga itu. Apa pun dapat terjadi pada bunga itu, semuanya dapat terjadi pada bunga, bunga terus-menerus dalam bahaya. Tapi benih itu aman; maka jutaan orang memutuskan untuk tetap menjadi benih. Tapi untuk tetap menjadi benih adalah tetap mati, tetap menjadi benih adalah tidak hidup sama sekali. Itu aman, tentu saja, tapi tidak memiliki kehidupan. Kematian itu aman, hidup adalah rasa tidak aman. Seseorang yang benar-benar ingin hidup harus hidup dalam bahaya, dalam bahaya terus-menerus. Seseorang yang ingin mencapai puncak harus mengambil risiko tersesat. Orang yang ingin mendaki puncak tertinggi harus mengambil risiko jatuh dari suatu tempat, tergelincir ke bawah.

\english
The seed cannot be destroyed so easily by the wind, the seed is very protected, secure. The flower is exposed -- such a delicate thing, and exposed to so many hazards: a strong wind may come, it may rain cats and dogs, the sun may be too hot, some foolish man may pluck the flower. Anything can happen to the flower, everything can happen to the flower, the flower is constantly in danger. But the seed is safe; hence millions of people decide to remain seeds. But to remain a seed is to remain dead, to remain a seed is not to live at all. It is secure, certainly, but it has no life. Death is secure, life is insecurity. One who really wants to live has to live in danger, in constant danger. One who wants to reach to the peaks has to take the risk of getting lost. One who wants to climb the highest peaks has to take the risk of falling from somewhere, slipping down.

\bahasa
Semakin besar kerinduan untuk tumbuh, semakin banyak dan semakin banyak bahaya yang harus diterima. Manusia sejati menerima bahaya sebagai gaya hidupnya, sebagai iklim pertumbuhannya.

\english
The greater is the longing to grow, the more and more danger has to be accepted. The real man accepts danger as his very style of life, as his very climate of growth.




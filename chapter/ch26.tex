\chapter{Elektron yang tidak logis} %334

\bahasa
8 Maret 1979 pagi di Aula Buddha

\english
8 March 1979 am in Buddha Hall

\bahasa
Pertanyaan pertama:

\english
The first question:

\bahasa
Osho Terkasih,\\
KEBEBASAN DAPAT DICAPAI MELALUI KEHENDAK, CINTA TIDAK. TOLONG BERI KOMENTAR.

\english
BELOVED OSHO,\\
FREEDOM CAN BE WILLED, LOVE NOT. PLEASE COMMENT.

\bahasa
Anand Akam, kebebasan dapat dicapai melalui kehendak karena kebebasan adalah keputusanmu sendiri untuk tetap berada di sebuah penjara. Itu adalah tanggung jawabmu sendiri. Engkau telah menghendaki perbudakanmu, engkau telah memutuskan untuk tetap menjadi seorang budak, maka engkau adalah seorang budak. Ubahlah keputusan itu, dan perbudakan lenyap.

\english
Anand Akam, freedom can be willed because it is your own decision to remain in a
prison. It is your own responsibility. You have willed your slavery, you have decided to remain a slave, hence you are a slave. Change the decision, and the slavery disappears.

\bahasa
Engkau telah berinvestasi dalam ketidakbebasanmu. Setiap momen engkau melihat intinya, engkau dapat menanggalkannya; itu langsung bisa ditanggalkan. Tidak ada yang memaksakan ketidakbebasan kepadamu, itu adalah pilihanmu. Engkau dapat memilih untuk bebas, engkau dapat memilih untuk tidak bebas; engkau sangat bebas sehingga engkau dapat memilih salah satunya. Ini adalah bagian dari kebebasan batinmu- tidak memilihnya adalah bagian dari kebebasanmu. Makanya itu dapat dicapai melalui kehendak.

\english
You have invested in your unfreedom. Any moment you see the point, you can drop it; instantly it can be dropped. Nobody has forced unfreedom on you, it is your choice. You can choose to be free, you can choose to be unfree; you are so free that you can choose either. This is part of your inner freedom -- not to choose it is part of your freedom. Hence it can be willed.

\bahasa
Tapi cinta tidak dapat dicapai melalui kehendak. Cinta adalah produk sampingan dari kebebasan; cinta adalah sukacita yang melimpah dari kebebasan, cinta adalah aroma kebebasan. Pertama, kebebasan harus ada di sana, lalu cinta mengikuti. Jika engkau mencoba untuk mencintai, engkau hanya akan menciptakan sesuatu yang tidaklah alami, buatan. Cinta yang dicapai melalui kehendak tidak akan menjadi cinta sejati, itu akan menjadi kepalsuan.

\english
But love cannot be willed. Love is a by-product of freedom; it is the overflowing joy of freedom, it is the fragrance of freedom. First the freedom has to be there, then love follows. If you try to will love, you will create only something artificial, arbitrary. A willed love will not be true love, it will be phony.

\bahasa
Dan itulah yang dilakukan orang-orang. Cinta tidak dapat dicapai melalui kehendak, dan mereka terus menghendakinya. Apa yang dapat dicapai melalui kehendak, kebebasan, mereka terus mengabaikannya. Mereka terus berpikir bahwa orang lain bertanggung jawab atas perbudakan dan kehidupan mereka atas perbudakan. Ini pemahaman yang sangat kacau balau akan hidupmu sendiri. Engkau terbalik.

\english
And that's what people are doing. Love cannot be willed, and they go on willing it. What can be willed, freedom, they go on ignoring. They go on thinking that somebody else is responsible for their slavery and their life of slavery. This is a very topsy-turvy conception of your own life. You are upside-down.

\bahasa
Ubahlah visi: Berkehendaklah untuk kebebasan, dan cinta akan datang dengan sendirinya. Ketika cinta datang dengan sendirinya maka itu indah, karena pada saat itulah cinta itu alami, spontan.

\english
Change the vision: will freedom, and love will come of its own accord. When love comes on its own accord only then it is beautiful, because only then it is natural, spontaneous.

\bahasa
Cinta yang dicapai melalui kehendak akan menjadi semacam akting. Engkau akan berpura-pura - apa lagi yang dapat engkau lakukan? Engkau akan bergerak melalui isyarat hampa - apa lagi yang mungkin? Engkau tidak dapat diperintahkan untuk mencintai seseorang, engkau tidak dapat memerintahkan dirimu sendiri untuk mencintai seseorang. Jika cinta itu tidak ada, itu tidak ada; jika cinta itu ada, itu ada disana. Cinta adalah sesuatu di luar kehendakmu. Sebenarnya cinta merupakan kebalikan dari kehendak: cinta adalah penyerahan diri.

\english
Willed love will be a kind of acting. You will be pretending -- what else can you do? You will be moving through empty gestures -- what else is possible? You cannot be ordered to love somebody, you cannot order yourself to love somebody. If it is not there, it is not there; if it is there, it is there. It is something beyond your will. In fact it is just the opposite of will: it is surrender.

\bahasa
Ketika seseorang benar-benar larut dalam kebebasan dan ketika seseorang benar-benar bebas, ego menghilang. Ego adalah ikatanmmu, ego adalah penjaramu. Didalam kebebasan yang total tidak ada ego yang ditemukan. Penyerahan-diri terjadi, engkau mulai merasakan satu dengan semesta - dan kesatuan itu membawa cinta.

\english
When one is totally dissolved into freedom and when one is really free, the ego disappears. The ego is your bondage, the ego is your prison. In total freedom there is no ego found. Surrender happens, you start feeling one with existence -- and that oneness brings love.


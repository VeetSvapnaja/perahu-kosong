\chapter{Manusia Tao}

\bahasa
MANUSIA TAO BERTINDAK TANPA KESUKARAN,

\english
THE MAN OF TAO ACTS WITHOUT IMPEDIMENT,

\bahasa
TIDAK MERUGIKAN ORANG LAIN DENGAN TINDAKANNYA,

\english
HARMS NO OTHER BEING BY HIS ACTIONS,

\bahasa
NAMUN DIA TIDAK TAHU DIRINYA BAIK DAN LEMBUT.

\english
YET HE DOES NOT KNOW HIMSELF TO BE KIND AND GENTLE.

\bahasa
DIA TIDAK BERGUMUL UNTUK MENGHASILKAN UANG,

\english
HE DOES NOT STRUGGLE TO MAKE MONEY,

\bahasa
DAN DIA TIDAK MEMBUAT SEBUAH KEBAJIKAN DARI KEMISKINAN.

\english
AND HE DOES NOT MAKE A VIRTUE OF POVERTY.

\bahasa
DIA PERGI KE JALANNYA TANPA BERGANTUNG PADA ORANG LAIN,

\english
HE GOES HIS WAY WITHOUT RELYING ON OTHERS,

\bahasa
DAN TIDAK MENYOMBONGKAN DIRI UNTUK BERJALAN SENDIRIAN

\english
AND DOES NOT PRIDE HIMSELF ON WALKING ALONE.

\bahasa
MANUSIA TAO TETAP TAK DIKENAL

\english
THE MAN OF TAO REMAINS UNKNOWN.

\bahasa
KEBAJIKAN SEMPURNA TIDAK MENGHASILKAN APAPUN.

\english
PERFECT VIRTUE PRODUCES NOTHING.

\bahasa
TIDAK ADA DIRI ADALAH DIRI SEJATI.

\english
NO SELF IS TRUE SELF.

\bahasa
DAN MANUSIA TERHEBAT ADALAH BUKAN SIAPA-SIAPA.

\english
AND THE GREATEST MAN IS NOBODY.

\bahasa
Hal yang paling sulit, hal yang hampir tidak mungkin bagi pikiran, adalah tetap berada di tengah, adalah tetap menjadi seimbang. Dan untuk bergerak dari satu hal ke hal yang sebaliknya adalah yang paling mudah. Untuk bergerak dari satu polaritas ke polaritas yang berlawanan adalah sifat alami dari pikiran. Ini harus dipahami sangat mendalam, karena kecuali engaku memahami hal ini, tidak ada yang dapat menuntunmu ke meditasi.

\english
The most difficult thing, the almost impossible thing for the mind, is to remain in the middle, is to remain balanced. And to move from one thing to its opposite is the easiest. To move from one polarity to the opposite polarity is the nature of the mind. This has to be understood very deeply, because unless you understand this, nothing can lead you to meditation.

\bahasa
Sifat alami pikiran adalah bergerak dari satu ekstrem ke yang lain. Itu tergantung pada ketidakseimbangan. Jika engkau seimbang, pikiran menghilang. Pikiran itu seperti sebuah penyakit: ketika engkau tidak seimbang pikiran ada di sana, ketika engkau seimbang, pikiran tidak ada disana.

\english
Mind's nature is to move from one extreme to another. It depends on imbalance. If you are balanced, mind disappears. Mind is like a disease: when you are imbalanced it is there, when you are balanced, it is not there.

\bahasa
Itulah mengapa mudah bagi seseorang yang makan terlalu banyak untuk berpuasa. Itu terlihat tidak masuk akal, karena kita berpikir bahwa orang yang terobsesi dengan makanan tidak bisa berpuasa. Tapi engkau salah. Hanya orang yang terobsesi dengan makanan dapat berpuasa, karena puasa adalah obsesi yang sama dalam arah yang berlawanan. Itu tidak benar-benar mengubah dirimu. Engkau masih terobsesi dengan makanan. Sebelumnya engkau makan berlebihan; sekarang engkau lapar -- tetapi pikiran tetap terfokus pada makanan dari ekstrem yang berlawanan

\english
That is why it is easy for a person who overeats to go on a fast. It looks illogical, because we think that a person who is obsessed with food cannot go on a fast. But you are wrong. Only a person who is obsessed with food can fast, because fasting is the same obsession in the opposite direction. It is not really changing yourself. You are still obsessed with food. Before you were overeating; now you are hungry -- but the mind remains focused on food from the opposite extreme

\bahasa
Seorang manusia yang telah terlalu banyak berhubungan seks dapat menjadi selibat dengan sangat mudah. Tidak ada masalah. Tetapi sulit bagi pikiran untuk tiba ke diet yang tepat, sulit bagi pikiran untuk tetap berada di tengah.

\english
A man who has been overindulging in sex can become a celibate very easily. There is no problem. But it is difficult for the mind to come to the right diet, difficult for the mind to stay in the middle.

\bahasa
Mengapa sulit untuk tetap berada di tengah? Analoginya seperti bandulan sebuah jam. Bandulan bergerak ke kanan, lalu bergerak ke kiri, lalu ke kanan, lalu ke kiri lagi; seluruh jam tergantung pada gerakan ini. Jika bandulan tetap di tengah, jam berhenti. Dan ketika bandulan bergerak ke kanan, engkau berpikir itu hanya akan bergerak ke kanan, tetapi pada saat yang sama itu mengumpulkan momentum untuk bergerak ke kiri. Semakin banyak bandulan itu bergerak ke kanan, semakin banyak energi yang dikumpulkannya untuk bergerak ke kiri, ke arah sebaliknya. Ketika bergerak ke kiri bandulan itu lagi-lagi mengumpulkan momentum untuk bergerak ke kanan.

\english
Why is it difficult to stay in the middle? It is just like the pendulum of a clock. The pendulum goes to the right, then it moves to the left, then again to the right, then again to the left; the whole clock depends on this movement. If the pendulum stays in the middle, the clock stops. And when the pendulum moves to the right, you think it is only going to the right, but at the same time it is gathering momentum to go to the left. The more it moves to the right, the more energy it gathers to move to the left, to the opposite. When it is moving to the left it is again gathering momentum to move to the right.

\bahasa
Setiap kali engkau makan berlebihan, engkau sedang mengumpulkan momentum untuk berpuasa. Setiap kali engkau berlebihan dalam seks, cepat atau lambat, BRAHMACHARYA, selibat, akan menarik bagimu.

\english
Whenever you overeat, you are gathering momentum to go on a fast. Whenever you overindulge in sex, sooner or later, BRAHMACHARYA, celibacy, will appeal to you.










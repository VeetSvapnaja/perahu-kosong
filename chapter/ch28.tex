\chapter{Jadilah Lelucon Untuk Dirimu Sendiri} %361

\bahasa
Pertanyaan pertama:

\english
The first question:

\bahasa
OSHO TERKASIH,\\
TOLONG JELASKAN PERBEDAAN ANTARA SANNYASIN DAN YANG TIDAK, NAMUN HIDUP DENGAN KOMITMEN YANG MENDALAM TERHADAP KEBENARAN.

\english
BELOVED OSHO,\\
PLEASE EXPLAIN THE DIFFERENCE BETWEEN A SANNYASIN AND ONE WHO IS NOT, YET LIVES WITH A DEEP COMMITMENT TO TRUTH.

\bahasa
Lynne Stevens, apakah engkau tahu apa itu kebenaran? Jika tidak, bagaimana dapat ada komitmen? Komitmen hanya mungkin jika engkau tahu. Sannyasin adalah orang yang tahu bahwa dia tidak tahu, sannyasin adalah orang yang komitmennya tidak untuk kebenaran, namun untuk penyelidikan kebenaran. Dan penyelidikannya adalah mungkin hanyia dengan seseorang yang tahu, seseorang yang telah tiba. Sannyasin adalah seseorang yang berkomitmen pada orang tersebut, atau bukan-pribadi, yang berada didekatnya dia merasakan getaran kebenaran, getaran keaslian.

\english
Lynne Stevens, do you know what truth is? Otherwise, how can there be a commitment? Commitment is possible only if you know. The sannyasin is one who knows that he knows not, the sannyasin is one whose commitment is not to truth but to the inquiry into truth. And the inquiry is possible only with someone who knows, who has arrived. The sannyasin is one who is committed to the person, or to the no-person, around whom he feels the vibe of truth, the vibe of authenticity.

\bahasa
Lynne Stevens, komitmenmu terhadap kebenaran hanyalah sebuah gagasan. Kebenaranmu hanyalah sebuah kata, sebuah perjalanan pikiran. Jika engkau ingin menjadikannya ziarah yang sebenarnya engkau harus menjadi murid - dan menjadi murid adalah menjadi sannyasin.

\english
Lynne Stevens, your commitment to truth is just an idea. Your truth is just a word, a mind trip. If you want to make it a real pilgrimage you will have to be a disciple -- and to be a disciple is to be a sannyasin.

\bahasa
Menjadi murid berarti siap untuk belajar, siap untuk masuk kedalam yang tidak diketahui dengan seseorang yang pernah berada di dalamnya. Sendiri, sangat jarang seseorang telah mencapai kebenaran. Bukan berarti hal itu tidak terjadi - sendiri, juga, hal itu telah terjadi, tapi sangat jarang, hanya pengecualian; Jika tidak, seseorang harus belajar dalam persekutuan dengan seorang master.

\english
To be a disciple means to be ready to learn, ready to go into the unknown with someone who has been in it. Alone, very rarely one has attained to truth. Not that it has not happened -- alone, also, it has happened, but very rarely, just an exception; otherwise one has to learn in communion with a master.

\bahasa
Lalu juga, itu tidak mudah terjadi. Itu adalah perjalanan yang sulit. Menjatuhkan kemelekatan terhadap yang diketahui tidaklah mudah. Itulah keseluruhan investasi kita, itulah keseluruhan identitas kita. Menjatuhkan kemelekatan yang diketahui adalah menjatuhkan ego, melakukan semacam bunuh diri spiritual; sendirian, engkau tidak akan dapat melakukannya. Kecuali engkau melihat seseorang yang telah melakukan bunuh diri itu dan masih ada - sebenarnya untuk pertama kalinya ada.... Engkau harus melihat ke dalam mata yang telah melihat kebenaran, dan sekilas kebenaran akan tertangkap melalui mata itu. . Engkau harus berpegangan tangan dengan seseorang yang telah mengetahui, menerima kehangatannya dan cintanya ... dan yang tidak diketahui akan mulai mengalir ke dalam dirimu.

\english
Then too, it does not happen easily. It is an arduous journey. Dropping the clinging to the known is not easy. That is our whole investment, that is our whole identity. Dropping the clinging to the known is dropping the ego, is committing a kind of spiritual suicide; alone, you will not be able to do it. Unless you see somebody who has committed that suicide and still is -- in fact for the first time is.... You will have to look into those eyes which have seen truth, and a glimpse of the truth will be caught through those eyes. You will have to hold hands with someone who has known, receive the warmth and the love... and the unknown will start flowing into you.

\bahasa
Itulah artinya bersama dengan seorang master, untuk menjadi murid. Jika engkau benar-benar berkomitmen terhadap kebenaran, engkau pasti akan menjadi sannyasin. Jika komitmenmu terhadap kebenaran adalah sebuah penyelidikan maka engkau harus mempelajari cara belajar. Dan hal pertama yang harus dipelajari adalah untuk berserah-diri, untuk percaya, untuk mencintai.

\english
That's what it means to be with a master, to be a disciple. If you are really committed to truth you are bound to become a sannyasin. If your commitment to truth is an inquiry then you will have to learn the ways of learning. And the first thing to learn is to surrender, to trust, to love.

\bahasa
Sannyasin adalah orang yang telah jatuh cinta pada seseorang, atau bukan seorang pribadi, di mana dia merasakan perasaan yang mendalam: "Ya, hal itu telah terjadi di sini." Untuk bersama seseorang yang dikenal menular - dan kebenaran tidak diajarkan, kebenaran itu ditangkap.

\english
The sannyasin is one who has fallen in love with a person, or a no-person, where he feels a gut feeling: "Yes, it has happened here." To be with someone who has known is contagious -- and truth is not taught, it is caught.

\bahasa
Kebenaranmu hanyalah sebuah gagasan di dalam pikiranmu - mungkin sebuah penyelidikan filosofis, tapi sebuah penyelidikan filosofis tidak akan membantu. Itu harus menjadi eksistensial, engkau harus memberikan bukti dalam hidupmu bahwa engkau benar-benar berkomitmen. Jika tidak, engkau dapat terus memainkan permainan kata-kata, permainan teori, sistem pemikiran yang indah - dan ada ribuan. Engkau juga dapat membuat sistem pemikiran pribadimu sendiri, dan engkau akan menganggap ini adalah kebenaran.

\english
Your truth is nothing but an idea in your mind -- maybe a philosophical inquiry, but a philosophical inquiry is not going to help. It has to become existential, you have to give proofs in your life that you are really committed. Otherwise you can go on playing the game of words, beautiful games of theories, systems of thought -- and there are thousands. You can also make a private system of thought of your own, and you will think this is truth.

\bahasa
Kebenaran bukanlah pembuatanmu, kebenaran tidak ada hubungannya dengan pikiranmu. Kebenaran terjadi, dan itu terjadi hanya jika engkau memiliki sebuah tanpa-pikiran. Tapi bagaimana engkau akan menjadi tanpa-pikiran? Dengan caramu sendiri engkau akan tetap menjadi si pikiran. Engkau mungkin berpikir tentang tanpa-pikiran, engkau dapat berfilsafat tentang tanpa-pikiran, engkau dapat membaca kitab suci tentang tanpa-pikiran, namun engkau akan tetap menjadi si pikiran. Dengan caramu sendiri, menyelidiki dan mencari, egomu akan merasa sangat senang - tapi itu adalah penghalang. Analoginya seperti menarik dirimu dengan benang kusutmu sendiri.

\english
Truth is not of your making, truth has nothing to do with your mind. Truth happens, and it happens only when you have become a no-mind. But how are you going to become a no-mind? On your own you will remain the mind. You may think about the no-mind, you may philosophize about the no-mind, you may read the scriptures about no-mind, but you will remain a mind. On your own, seeking and searching, your ego will feel very good -- but that is the barrier. It is like pulling yourself up by your own bootstraps.

\bahasa
Jika di suatu tempat engkau menemukan bantuan tersedia, jangan lewatkan itu - karena kesempatan itu langka, buddhafield jarang terjadi. Hanya sekali-sekali, entah di mana, seorang buddha muncul, bodhichitta terjadi. Maka jangan lewatkan kesempatan itu. Jika komitmenmu benar-benar menuju pada kebenaran, engkau tidak dapat menghindari menjadi sannyasin. Hal ini tak terelakkan, karena tanpa pikiran dipelajari hanya dengan duduk di sisi seorang tanpa pikiran.

\english
If somewhere you find help is available, don't miss it -- because the opportunity is rare, the buddhafield is rare. Only once in a while, somewhere, a buddha arises, a bodhichitta happens. Then don't miss the opportunity. If your commitment is really towards truth, you cannot avoid becoming a sannyasin. It is inevitable, because no-mind is learned only by sitting by the side of a no-mind.

\bahasa
Jika engkau duduk di sisiku, perlahan pelan-pelan pikiranmu akan mulai menghilang seperti kabut pagi. Perlahan perlahan keheningan akan mulai menembusmu -- bukan tindakanmu, tapi datang dengan sendirinya. Keheningan akan menyelimutimu.

\english
If you sit by my side, slowly slowly your mind will start dispersing like the morning mist. Slowly slowly a silence will start penetrating you -- not of your doing, but coming on its own. A stillness will pervade you.

\bahasa
Dan saat engkau benar-benar diam, bahkan pikiran yang bergerak di dalam dirimu, itulah momen iluminasi Untuk pertama kalinya engkau memiliki sekilas kebenaran - bukan gagasan akan kebenaran, tapi kebenaran itu sendiri

\english
And the moment you are utterly still, not even a thought moving inside you, that is the moment of illumination. For the first time you have a glimpse of truth -- not the idea of truth, but truth itself.

\bahasa
Pertanyaan Kedua:

\english
The second question:

\bahasa
OSHO Terkasih,\\
PERASAANKU MENGATAKAN BAHWA SAMPAI AKU MENGENALMU, AKU TIDAK DAPAT PERCAYA, NAMUN ENGKAU MENGATAKAN SAMPAI AKU MEMPERCAYAIMU, AKU TIDAK DAPAT MENGENALMU. APA YANG HARUS DILAKUKAN?

\english
BELOVED OSHO,\\
MY FEELINGS TELL ME THAT UNTIL I KNOW YOU, I CAN'T TRUST. AND YET YOU SAY UNTIL I TRUST YOU, I CANNOT KNOW YOU. WHAT TO DO?

\bahasa
William, ada dua jenis pengetahuan. Yang satu adalah dari kejauhan: engkau tetap menyendiri, engkau tetap menjadi seorang pengamat, penonton. Itulah pengetahuan secara ilmiah; engkau tidak perlu terlibat di dalamnya, sebenarnya engkau seharusnya tidak terlibat. Engkau harus sangat objektif, engkau seharusnuya tidak membiarkan subjektivitasmu mencampuri pengamatanmu. Engkau seharusnya berada di sana seperti seorang pengamat mekanis. Engkau seharusnya tidak menjadi manusia, engkau seharusnya hanya menjadi sebuah komputer.

\english
William, there are two kinds of knowing. One is from a distance: you remain aloof, you remain an observer, a spectator. That's what scientific knowing is; you need not get involved in it, in fact you should not get involved. You should be very objective, you should not allow your subjectivity to interfere with your observation. You should simply be there like a mechanical watcher. You should not be a human being, you should be just a computer.

\bahasa
Dan ini pasti, bahwa cepat atau lambat sains akan diambil alih oleh komputer, robot, karena mereka akan lebih ilmiah. Tidak akan ada subjektivitas di dalamnya, mereka hanya akan melihat fakta. Fakta tidak akan dicampuri dengan cara apapun, itu akan tetap benar-benar objektif.

\english
And this is certain, that sooner or later science is going to be taken over by computers, robots, because they will be more scientific. There will be no subjectivity in them, they will simply see the fact. The fact will not be interfered with in any way, it will remain utterly objective.

\bahasa
Itulah cara sains - mengetahui dari kejauhan, menjaga jarak, terpisah. Begitulah cara ilmuwan akan mengetahui sekuntum bunga mawar, begitulah cara ilmuwan akan mengetahui matahari terbenam, begitulah cara ilmuwan akan mengetahui keindahan seorang wanita atau seorang pria.

\english
That is the way of science -- knowing from a distance, keeping aloof, detached. That's how the scientist will know a rose flower, that's how the scientist will know the sunset, that's how the scientist will know the beauty of a woman or a man.

\bahasa
Tapi masalahnya adalah, sesuatu yang esensial pasti akan dilewatkan, sesuatu yang sangat fundamental, sesuatu yang merupakan inti dari semuanya. Ilmuwan dapat mengetahui bunga mawar - dia dapat tahu itu terbentuk dari apa, dia dapat mengetahui semua bahan kimia,dan sebagainya, tapi dia tidak akan pernah tahu keindahannya. Dia akan tetap buta terhadap keindahan; Pendekatannya, metodologinya, menghalanginya.

\english
But the problem is, something essential is bound to be missed, something very
fundamental, something which is the core of the whole thing. The scientist can know the roseflower -- he can know of what it is constituted, he can know all the chemicals, etcetera, but he will never know the beauty of it. He will remain blind to the beauty; his very approach, his methodology, prohibits him.

\bahasa
Jika engkau terpisah engkau tidak dapat mengenal keindahan. Keindahan hanya diketahui saat engkau jatuh dalam hubungan yang erat, saat pengamat menjadi yang teramati, bila tidak ada dinding di antara, ketika setiap dinding telah berubah menjadi jembatan. Bila ada semacam pencairan, ketika engkau menjadi bunga dan bunga menjadi dirimu, maka ada jenis pengetahuan yang sama sekali berbeda - seperti yang diketahui seorang penyair. Dia akan tahu kecantikan, dia tidak akan tahu bahan kimianya. Dia tidak akan tahu bunga yang objektif, dia akan tahu sesuatu yang jauh lebih dalam. Dia akan mengetahui spiritualitas dari bunga, jiwa dari bunga.

\english
If you are detached you cannot know beauty. Beauty is known only when you fall en rapport, when the observer becomes the observed, when there is no wall between, when every wall has been transformed into a bridge. When there is a kind of melting, when you become the flower and the flower becomes you, then there is a totally different kind of knowing -- the way a poet knows. He will know beauty, he will not know the chemicals. He will not know the objective flower, he will know something far deeper. He will know the spirituality of the flower, the spirit of the flower.

\bahasa
Dan para mistik, pengetahuannya adalah bentuk tertinggi dari pengetahuan puitis, bentuk akhir dari pengetahuan puitis. Penyair hanya ada untuk beberapa saat. Terkadang dia seorang penyair, dia bertemu, dia bercampur, menyatu menjadi bunga; terkadang ia menjadi pengamat yang terpisah. Oleh karena itu puisi adalah semacam campuran dari kedua pengetahuan tersebut.

\english
And the mystic, his knowing is the highest form of poetic knowing, the ultimate form of poetic knowing. The poet is there only for moments. Sometimes he is a poet, he meets, he mingles, merges into the flower; sometimes he becomes a detached observer. Hence poetry is a kind of mixture of both the knowledges.

\bahasa

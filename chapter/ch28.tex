\chapter{Jadilah Lelucon Untuk Dirimu Sendiri} %361

\bahasa
Pertanyaan pertama:

\english
The first question:

\bahasa
OSHO TERKASIH,\\
TOLONG JELASKAN PERBEDAAN ANTARA SANNYASIN DAN YANG TIDAK, NAMUN HIDUP DENGAN KOMITMEN YANG MENDALAM TERHADAP KEBENARAN.

\english
PLEASE EXPLAIN THE DIFFERENCE BETWEEN A SANNYASIN AND ONE WHO IS NOT, YET LIVES WITH A DEEP COMMITMENT TO TRUTH.

\bahasa
Lynne Stevens, apakah engkau tahu apa itu kebenaran? Jika tidak, bagaimana dapat ada komitmen? Komitmen hanya mungkin jika engkau tahu. Sannyasin adalah orang yang tahu bahwa dia tidak tahu, sannyasin adalah orang yang komitmennya tidak untuk kebenaran, namun untuk penyelidikan kebenaran. Dan penyelidikannya adalah mungkin hanyia dengan seseorang yang tahu, seseorang yang telah tiba. Sannyasin adalah seseorang yang berkomitmen pada orang tersebut, atau bukan-pribadi, yang berada didekatnya dia merasakan getaran kebenaran, getaran keaslian.

\english
Lynne Stevens, do you know what truth is? Otherwise, how can there be a commitment? Commitment is possible only if you know. The sannyasin is one who knows that he knows not, the sannyasin is one whose commitment is not to truth but to the inquiry into truth. And the inquiry is possible only with someone who knows, who has arrived. The sannyasin is one who is committed to the person, or to the no-person, around whom he feels the vibe of truth, the vibe of authenticity.

\bahasa
Lynne Stevens, komitmenmu terhadap kebenaran hanyalah sebuah gagasan. Kebenaranmu hanyalah sebuah kata, sebuah perjalanan pikiran. Jika engkau ingin menjadikannya ziarah yang sebenarnya engkau harus menjadi murid - dan menjadi murid adalah menjadi sannyasin.

\english
Lynne Stevens, your commitment to truth is just an idea. Your truth is just a word, a mind trip. If you want to make it a real pilgrimage you will have to be a disciple -- and to be a disciple is to be a sannyasin.

\bahasa
Menjadi murid berarti siap untuk belajar, siap untuk masuk kedalam yang tidak diketahui dengan seseorang yang pernah berada di dalamnya. Sendiri, sangat jarang seseorang telah mencapai kebenaran. Bukan berarti hal itu tidak terjadi - sendiri, juga, hal itu telah terjadi, tapi sangat jarang, hanya pengecualian; Jika tidak, seseorang harus belajar dalam persekutuan dengan seorang master.

\english
To be a disciple means to be ready to learn, ready to go into the unknown with someone who has been in it. Alone, very rarely one has attained to truth. Not that it has not happened -- alone, also, it has happened, but very rarely, just an exception; otherwise one has to learn in communion with a master.

\bahasa
Lalu juga, itu tidak mudah terjadi. Itu adalah perjalanan yang sulit. Menjatuhkan kemelekatan terhadap yang diketahui tidaklah mudah. Itulah keseluruhan investasi kita, itulah keseluruhan identitas kita. Menjatuhkan kemelekatan yang diketahui adalah menjatuhkan ego, melakukan semacam bunuh diri spiritual; sendirian, engkau tidak akan dapat melakukannya. Kecuali engkau melihat seseorang yang telah melakukan bunuh diri itu dan masih ada - sebenarnya untuk pertama kalinya ada.... Engkau harus melihat ke dalam mata yang telah melihat kebenaran, dan sekilas kebenaran akan tertangkap melalui mata itu. . Engkau harus berpegangan tangan dengan seseorang yang telah mengetahui, menerima kehangatannya dan cintanya ... dan yang tidak diketahui akan mulai mengalir ke dirimu.

\english
Then too, it does not happen easily. It is an arduous journey. Dropping the clinging to the known is not easy. That is our whole investment, that is our whole identity. Dropping the clinging to the known is dropping the ego, is committing a kind of spiritual suicide; alone, you will not be able to do it. Unless you see somebody who has committed that suicide and still is -- in fact for the first time is.... You will have to look into those eyes which have seen truth, and a glimpse of the truth will be caught through those eyes. You will have to hold hands with someone who has known, receive the warmth and the love... and the unknown will start flowing into you.

\bahasa
Itulah artinya bersama dengan seorang master, untuk menjadi murid. Jika engkau benar-benar berkomitmen terhadap kebenaran, engkau pasti akan menjadi sannyasin. Jika komitmenmu terhadap kebenaran adalah sebuah penyelidikan maka engkau harus mempelajari cara belajar. Dan hal pertama yang harus dipelajari adalah untuk berserah-diri, untuk percaya, untuk mencintai.

\english
That's what it means to be with a master, to be a disciple. If you are really committed to truth you are bound to become a sannyasin. If your commitment to truth is an inquiry then you will have to learn the ways of learning. And the first thing to learn is to surrender, to trust, to love.

\bahasa
Sannyasin adalah orang yang telah jatuh cinta pada seseorang, atau bukan seorang pribadi, di mana dia merasakan perasaan yang mendalam: "Ya, hal itu telah terjadi di sini." Untuk bersama seseorang yang dikenal menular - dan kebenaran tidak diajarkan, kebenaran itu ditangkap.

\english
The sannyasin is one who has fallen in love with a person, or a no-person, where he feels a gut feeling: "Yes, it has happened here." To be with someone who has known is contagious -- and truth is not taught, it is caught.

\bahasa
Kebenaranmu hanyalah sebuah gagasan di dalam pikiranmu - mungkin sebuah penyelidikan filosofis, tapi sebuah penyelidikan filosofis tidak akan membantu. Itu harus menjadi eksistensial, engkau harus memberikan bukti dalam hidupmu bahwa engkau benar-benar berkomitmen. Jika tidak, engkau dapat terus memainkan permainan kata-kata, permainan teori, sistem pemikiran yang indah - dan ada ribuan. Engkau juga dapat membuat sistem pemikiran pribadimu sendiri, dan engkau akan menganggap ini adalah kebenaran.

\english
Your truth is nothing but an idea in your mind -- maybe a philosophical inquiry, but a philosophical inquiry is not going to help. It has to become existential, you have to give proofs in your life that you are really committed. Otherwise you can go on playing the game of words, beautiful games of theories, systems of thought -- and there are thousands. You can also make a private system of thought of your own, and you will think this is truth.

\bahasa
Kebenaran bukanlah pembuatanmu, kebenaran tidak ada hubungannya dengan pikiranmu. Kebenaran terjadi, dan itu terjadi hanya jika engkau memiliki sebuah tanpa-pikiran. Tapi bagaimana engkau akan menjadi tanpa-pikiran? Dengan caramu sendiri engkau akan tetap menjadi si pikiran. Engkau mungkin berpikir tentang tanpa-pikiran, engkau dapat berfilsafat tentang tanpa-pikiran, engkau dapat membaca kitab suci tentang tanpa-pikiran, namun engkau akan tetap menjadi si pikiran. Dengan caramu sendiri, menyelidiki dan mencari, egomu akan merasa sangat senang - tapi itu adalah penghalang. Analoginya seperti menarik dirimu dengan benang kusutmu sendiri.

\english
Truth is not of your making, truth has nothing to do with your mind. Truth happens, and it happens only when you have become a no-mind. But how are you going to become a no-mind? On your own you will remain the mind. You may think about the no-mind, you may philosophize about the no-mind, you may read the scriptures about no-mind, but you will remain a mind. On your own, seeking and searching, your ego will feel very good -- but that is the barrier. It is like pulling yourself up by your own bootstraps.

